%% Generated by Sphinx.
\def\sphinxdocclass{report}
\documentclass[letterpaper,10pt,spanish]{sphinxmanual}
\ifdefined\pdfpxdimen
   \let\sphinxpxdimen\pdfpxdimen\else\newdimen\sphinxpxdimen
\fi \sphinxpxdimen=.75bp\relax

\PassOptionsToPackage{warn}{textcomp}
\usepackage[utf8]{inputenc}
\ifdefined\DeclareUnicodeCharacter
% support both utf8 and utf8x syntaxes
  \ifdefined\DeclareUnicodeCharacterAsOptional
    \def\sphinxDUC#1{\DeclareUnicodeCharacter{"#1}}
  \else
    \let\sphinxDUC\DeclareUnicodeCharacter
  \fi
  \sphinxDUC{00A0}{\nobreakspace}
  \sphinxDUC{2500}{\sphinxunichar{2500}}
  \sphinxDUC{2502}{\sphinxunichar{2502}}
  \sphinxDUC{2514}{\sphinxunichar{2514}}
  \sphinxDUC{251C}{\sphinxunichar{251C}}
  \sphinxDUC{2572}{\textbackslash}
\fi
\usepackage{cmap}
\usepackage[T1]{fontenc}
\usepackage{amsmath,amssymb,amstext}
\usepackage{babel}



\usepackage{times}
\expandafter\ifx\csname T@LGR\endcsname\relax
\else
% LGR was declared as font encoding
  \substitutefont{LGR}{\rmdefault}{cmr}
  \substitutefont{LGR}{\sfdefault}{cmss}
  \substitutefont{LGR}{\ttdefault}{cmtt}
\fi
\expandafter\ifx\csname T@X2\endcsname\relax
  \expandafter\ifx\csname T@T2A\endcsname\relax
  \else
  % T2A was declared as font encoding
    \substitutefont{T2A}{\rmdefault}{cmr}
    \substitutefont{T2A}{\sfdefault}{cmss}
    \substitutefont{T2A}{\ttdefault}{cmtt}
  \fi
\else
% X2 was declared as font encoding
  \substitutefont{X2}{\rmdefault}{cmr}
  \substitutefont{X2}{\sfdefault}{cmss}
  \substitutefont{X2}{\ttdefault}{cmtt}
\fi


\usepackage[Sonny]{fncychap}
\ChNameVar{\Large\normalfont\sffamily}
\ChTitleVar{\Large\normalfont\sffamily}
\usepackage{sphinx}

\fvset{fontsize=\small}
\usepackage{geometry}


% Include hyperref last.
\usepackage{hyperref}
% Fix anchor placement for figures with captions.
\usepackage{hypcap}% it must be loaded after hyperref.
% Set up styles of URL: it should be placed after hyperref.
\urlstyle{same}

\addto\captionsspanish{\renewcommand{\contentsname}{Contenidos:}}

\usepackage{sphinxmessages}
\setcounter{tocdepth}{1}



\title{Atencion\sphinxhyphen{}1 Manual de Usuarios}
\date{15 de junio de 2020}
\release{june,2020}
\author{qu4nt}
\newcommand{\sphinxlogo}{\vbox{}}
\renewcommand{\releasename}{Versión}
\makeindex
\begin{document}

\ifdefined\shorthandoff
  \ifnum\catcode`\=\string=\active\shorthandoff{=}\fi
  \ifnum\catcode`\"=\active\shorthandoff{"}\fi
\fi

\pagestyle{empty}
\sphinxmaketitle
\pagestyle{plain}
\sphinxtableofcontents
\pagestyle{normal}
\phantomsection\label{\detokenize{index::doc}}
\noindent\sphinxincludegraphics{{symbol_grupov_alt}.png}




\chapter{v. 0.1}
\label{\detokenize{index:v-0-1}}

\chapter{Junio, 2020}
\label{\detokenize{index:junio-2020}}

\section{qu4nt}
\label{\detokenize{index:qu4nt}}

\subsection{Introducción}
\label{\detokenize{introduccion:introduccion}}\label{\detokenize{introduccion::doc}}

\subsection{Glosario}
\label{\detokenize{glosario:glosario}}\label{\detokenize{glosario::doc}}\begin{itemize}
\item {} 
\sphinxstylestrong{Administrador}: Rol de usuario del Sistema Atención\sphinxhyphen{}1, con permisos de acceso a la interfaz administrativa de la plataforma.

\item {} 
\sphinxstylestrong{Afiliado}: Persona que contrata o es beneficiario de una póliza de seguros. Atención\sphinxhyphen{}1 maneja dos tipos de afiliados: Titular (persona a nombre de quien figura la póliza), y Beneficiario/a (pesona que se incluye en la cobertura de la póliza)

\item {} 
\sphinxstylestrong{AMD}:Abreviatura de Atención Médica Domiciliaria, que es uno de los servicios gestionados por Atención\sphinxhyphen{}1 a través del cual un afiliado recibe servicios sanitarios a través de un médico en un lugar distinto de un centro médico.

\item {} 
\sphinxstylestrong{Atención}: Conjunto de servicios involucrados en una solución médica ofrecida a afiliados de las organizaciones aliadas de Atención\sphinxhyphen{}1.

\item {} 
\sphinxstylestrong{Base de datos}: Estructura digital que permite el registro, almacenamiento, acceso, modificación y eliminación de información y datos.

\item {} 
\sphinxstylestrong{Cliente}: Organizaciones, públicas o privadas, que prestan sus servicios de atención médica a través de Atención\sphinxhyphen{}1.

\item {} 
\sphinxstylestrong{Contratante}: Organización o persona que adquiere una póliza a través de cualquiera de las organizaciones aliadas de Atención\sphinxhyphen{}1.

\item {} 
\sphinxstylestrong{Coordinador}: Rol de usuario del Sistema Atención\sphinxhyphen{}1, con permisos de acceso a tareas de gestión de servicios y atenciones dentro de la plataforma Atención\sphinxhyphen{}1.

\item {} 
\sphinxstylestrong{Despachador}: Rol de usuario del Sistema Atención\sphinxhyphen{}1, con permisos de acceso a labores de gestión de tripulaciones, operaciones con traslados de afiliados, entregas de medicamentos y otros servicios ofrecidos por Atención\sphinxhyphen{}1 que involucren el traslado de medicamentos, personal y/o equipos.

\item {} 
\sphinxstylestrong{Director}: Rol de usuario del Atención\sphinxhyphen{}1, con permisos de acceso a la gestión integral de todas las operaciones de Atención\sphinxhyphen{}1, desde la creación de atenciones, hasta el cierre y cancelación de servicios y atenciones y la generación de reportes.

\item {} 
\sphinxstylestrong{Destino}: Dirección en la que se prevé termine un servicio que involucre un traslado o el desplazamiento de una unidad vehicular.

\item {} 
\sphinxstylestrong{Diagnóstico}: Procedimiento médico a través del cual se identifica cualquier enfermedad, sindrome o similar.

\item {} 
\sphinxstylestrong{EMD}: Abreviatura de Entrega de Medicamentos Domiciliaria, que es parte de los servicios gestionados por Atención\sphinxhyphen{}1 y que permite el envío de medicamentos solicitados por cualquier afiliado de pólizas con estos beneficios, al domicilio que éste indique.

\item {} 
\sphinxstylestrong{Gerente}: Rol de usuario de Atención\sphinxhyphen{}1, con permisos de acceso a la gestión de las operaciones y supervisión en el mismo. Este rol es el encargado de la solicitud y generación de reportes sobre estatus en tiempo real de las atenciones y servicios y creación de tripulación entre otros.

\item {} 
\sphinxstylestrong{Indicaciones}: Conjunto de instrucciones detalladas sobre medicamentos indicados según diagnóstico, y presentadas en un récipe médico a la persona afiliada con la información necesaria para la realización del tratamiento médico indicado por el personal médico.

\item {} 
\sphinxstylestrong{Médico}: Rol deusuario de Atención\sphinxhyphen{}1, con permisos de acceso

\item {} 
\sphinxstylestrong{Medicamento}: Sustancia que es utilizada en la prevención o tratamiento de alguna enfermedad, bien sea eliminando sus efectos en el organismo, o mitigándolos.

\item {} 
\sphinxstylestrong{OMT}: Abreviatura de Orientación Médica Telefónica, que es parte de los servicios gestionados por Atención\sphinxhyphen{}1 y que implica la atención telefónica de afiliados en la solución de sus consultas o derivación hacia otros servicios.

\item {} 
\sphinxstylestrong{Organización}: Entidad colectiva, de tipo Contratante, Cliente o Externo, que presta o recibe servicios de salud a través de la plataforma Atención\sphinxhyphen{}1.

\item {} 
\sphinxstylestrong{Proveedor Externo}: Empresas de traslados y servicios médicos que sirve de apoyo en los servicios AMD y EMD cuando éstos se prestan fuera del ámbito de la Gran Caracas.

\item {} 
\sphinxstylestrong{Operador}: Rol de usuario del Sistema Atención\sphinxhyphen{}1, con permisos de acceso a tareas de toma y registro de atenciones y solicitudes de servicios dentro de la plataforma Atención\sphinxhyphen{}1.

\item {} 
\sphinxstylestrong{Plan}: Características relativas a tipo de servicios disponibles

\item {} 
\sphinxstylestrong{Póliza}: Número que identifica al documento que recoge las condiciones generales del seguro contratado, y las condiciones particulares así como los suplementos o apéndices emitidos para complementarla o modificarla.

\item {} 
\sphinxstylestrong{Reporte}

\item {} 
\sphinxstylestrong{Servicio}

\item {} 
\sphinxstylestrong{Síntoma}

\item {} 
\sphinxstylestrong{TLD}

\item {} 
\sphinxstylestrong{Tripulación}

\item {} 
\sphinxstylestrong{Usuario}

\end{itemize}
\begin{quote}

t\_tqRYkQkvoiTAit\_h4v
\end{quote}


\subsection{Descripción de Atención\sphinxhyphen{}1}
\label{\detokenize{descripcion_software:descripcion-de-atencion-1}}\label{\detokenize{descripcion_software::doc}}

\subsection{Funciones y roles en Atención\sphinxhyphen{}1}
\label{\detokenize{roles_funciones:funciones-y-roles-en-atencion-1}}\label{\detokenize{roles_funciones::doc}}

\subsection{Buenas prácticas}
\label{\detokenize{buenas_pr_xe1cticas:buenas-practicas}}\label{\detokenize{buenas_pr_xe1cticas::doc}}

\subsection{Acerca de Atención\sphinxhyphen{}1}
\label{\detokenize{about:acerca-de-atencion-1}}\label{\detokenize{about::doc}}

\chapter{Indices y tablas}
\label{\detokenize{index:indices-y-tablas}}\begin{itemize}
\item {} 
\DUrole{xref,std,std-ref}{genindex}

\item {} 
\DUrole{xref,std,std-ref}{modindex}

\item {} 
\DUrole{xref,std,std-ref}{search}

\end{itemize}



\renewcommand{\indexname}{Índice}
\printindex
\end{document}