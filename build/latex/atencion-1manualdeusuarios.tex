%% Generated by Sphinx.
\def\sphinxdocclass{report}
\documentclass[letterpaper,10pt,spanish]{sphinxmanual}
\ifdefined\pdfpxdimen
   \let\sphinxpxdimen\pdfpxdimen\else\newdimen\sphinxpxdimen
\fi \sphinxpxdimen=.75bp\relax

\PassOptionsToPackage{warn}{textcomp}
\usepackage[utf8]{inputenc}
\ifdefined\DeclareUnicodeCharacter
% support both utf8 and utf8x syntaxes
  \ifdefined\DeclareUnicodeCharacterAsOptional
    \def\sphinxDUC#1{\DeclareUnicodeCharacter{"#1}}
  \else
    \let\sphinxDUC\DeclareUnicodeCharacter
  \fi
  \sphinxDUC{00A0}{\nobreakspace}
  \sphinxDUC{2500}{\sphinxunichar{2500}}
  \sphinxDUC{2502}{\sphinxunichar{2502}}
  \sphinxDUC{2514}{\sphinxunichar{2514}}
  \sphinxDUC{251C}{\sphinxunichar{251C}}
  \sphinxDUC{2572}{\textbackslash}
\fi
\usepackage{cmap}
\usepackage[T1]{fontenc}
\usepackage{amsmath,amssymb,amstext}
\usepackage{babel}



\usepackage{times}
\expandafter\ifx\csname T@LGR\endcsname\relax
\else
% LGR was declared as font encoding
  \substitutefont{LGR}{\rmdefault}{cmr}
  \substitutefont{LGR}{\sfdefault}{cmss}
  \substitutefont{LGR}{\ttdefault}{cmtt}
\fi
\expandafter\ifx\csname T@X2\endcsname\relax
  \expandafter\ifx\csname T@T2A\endcsname\relax
  \else
  % T2A was declared as font encoding
    \substitutefont{T2A}{\rmdefault}{cmr}
    \substitutefont{T2A}{\sfdefault}{cmss}
    \substitutefont{T2A}{\ttdefault}{cmtt}
  \fi
\else
% X2 was declared as font encoding
  \substitutefont{X2}{\rmdefault}{cmr}
  \substitutefont{X2}{\sfdefault}{cmss}
  \substitutefont{X2}{\ttdefault}{cmtt}
\fi


\usepackage[Sonny]{fncychap}
\ChNameVar{\Large\normalfont\sffamily}
\ChTitleVar{\Large\normalfont\sffamily}
\usepackage{sphinx}

\fvset{fontsize=\small}
\usepackage{geometry}


% Include hyperref last.
\usepackage{hyperref}
% Fix anchor placement for figures with captions.
\usepackage{hypcap}% it must be loaded after hyperref.
% Set up styles of URL: it should be placed after hyperref.
\urlstyle{same}

\addto\captionsspanish{\renewcommand{\contentsname}{Contenidos:}}

\usepackage{sphinxmessages}
\setcounter{tocdepth}{1}



\title{Atencion\sphinxhyphen{}1 Manual de Usuarios}
\date{25 de junio de 2020}
\release{junio,2020}
\author{qu4nt}
\newcommand{\sphinxlogo}{\vbox{}}
\renewcommand{\releasename}{Versión}
\makeindex
\begin{document}

\ifdefined\shorthandoff
  \ifnum\catcode`\=\string=\active\shorthandoff{=}\fi
  \ifnum\catcode`\"=\active\shorthandoff{"}\fi
\fi

\pagestyle{empty}
\sphinxmaketitle
\pagestyle{plain}
\sphinxtableofcontents
\pagestyle{normal}
\phantomsection\label{\detokenize{index::doc}}
\noindent\sphinxincludegraphics{{symbol_grupov_alt}.png}




\chapter{Descripción de Atención\sphinxhyphen{}1}
\label{\detokenize{01descripcion_software:descripcion-de-atencion-1}}\label{\detokenize{01descripcion_software::doc}}
Atención\sphinxhyphen{}1 es una aplicación web del Grupo Venemergencia que tiene como finalidad gestionar los siguientes servicios de atención y emergencias médicas: Orientación Médica Telefónica (OMT), Atención Médica Domiciliaria (AMD), Traslados en Ambulancia, entre otros.

La aplicación permite el registro de contacto y datos personales del paciente, los servicios requeridos de cada atención y su seguimiento, lo que ayudará a generar los datos necesarios para la creación de informes estadísticos.

La solicitud del servicio comienza a partir del primer contacto del paciente, ya sea por medio de una llamada telefónica u otro canal establecido. Para que dicho paciente pueda gozar de este servicio debe estar afiliado a un plan de póliza de atención médica.

Por medio de un operador se crea la atención y verifican los datos del paciente para luego comunicarse con el médico correspondiente que sería el responsable de proporcionar el tratamiento a partir de los síntomas y observaciones. El tratamiento comprende récipe, indicaciones y la orden de exámenes médicos. También se brindan servicios externos de atención que incluyen atención médica domiciliaria, entrega de medicamentos y servicios de traslado, pero estos servicios dependen, una vez más, del plan de póliza del paciente. Por ello, la organización de la tripulación comprendida por los médicos, paramédicos y choferes deben estar organizados para el cumplimiento de sus guardias las cuales se encuentran bajo la responsabilidad del despachador.

Grupo Venemergencia ofrece distintos servicios de atención médica como traslados, consultas a domicilio, despacho de medicamentos, entre otros, cuya información será manejada en su totalidad a través de la aplicación Atención\sphinxhyphen{}1.


\chapter{Funciones y roles en Atención\sphinxhyphen{}1}
\label{\detokenize{02roles_funciones:funciones-y-roles-en-atencion-1}}\label{\detokenize{02roles_funciones::doc}}
Atención\sphinxhyphen{}1 organiza su funcionamiento a través de cinco roles de usuarios principales. Un \sphinxstylestrong{rol} es el conjunto de permisos y atribuciones con los que cuenta un usuario del sistema y que le permite al usuario que cuenta con él, realizar determinadas funciones dentro de la plataforma.

Si quieres conocer otros conceptos de la plataforma Atención\sphinxhyphen{}1, te sugerimos revisar este {\hyperref[\detokenize{05glosario:glosario}]{\sphinxcrossref{\DUrole{std,std-ref}{Glosario}}}} que hemos preparado.


\section{Operador/a}
\label{\detokenize{01operador:operador-a}}\label{\detokenize{01operador::doc}}
El Operador es la persona que atiende la llamada telefónica y agrega la información del afiliado y su solicitud en la plataforma Atención\sphinxhyphen{}1. Recuerda que si quieres conocer los conceptos utilizados en la plataforma Atención\sphinxhyphen{}1, puedes revisar este {\hyperref[\detokenize{05glosario:glosario}]{\sphinxcrossref{\DUrole{std,std-ref}{Glosario}}}} que hemos preparado.

Para ingresar a la plataforma como Operador, debemos escribir en la barra de dirección de nuestro navegador: \sphinxurl{https://atencion1.venedigital.com}. Una vez allí ingresaremos nuestro usuario y contraseña para ingresar a la plataforma.

\noindent\sphinxincludegraphics{{PantallaInicialAtencion1}.png}

Una vez dentro de la plataforma, vemos el panel de atenciones abiertas, en progreso y por cerrar. Para crear una nueva \sphinxstylestrong{atención} pinchamos en el círculo de color verde con un signo «+» en su interior.


\section{Medico/a}
\label{\detokenize{02medico:medico-a}}\label{\detokenize{02medico::doc}}

\section{Despachador/a}
\label{\detokenize{03despachador:despachador-a}}\label{\detokenize{03despachador::doc}}

\section{Coordinador/a}
\label{\detokenize{04coordinador:coordinador-a}}\label{\detokenize{04coordinador::doc}}

\section{Asistente}
\label{\detokenize{05asistente:asistente}}\label{\detokenize{05asistente::doc}}

\section{Gerente}
\label{\detokenize{06gerente:gerente}}\label{\detokenize{06gerente::doc}}

\section{Director/a}
\label{\detokenize{07director:director-a}}\label{\detokenize{07director::doc}}

\chapter{Buenas prácticas}
\label{\detokenize{03buenas_pr_xe1cticas:buenas-practicas}}\label{\detokenize{03buenas_pr_xe1cticas::doc}}

\chapter{Permisos según roles en Atención\sphinxhyphen{}1}
\label{\detokenize{04permisos:permisos-segun-roles-en-atencion-1}}\label{\detokenize{04permisos::doc}}
La siguiente tabla resume los distintos permisos de creación, modificación, eliminación y consulta de atenciones, servicios, afiliados, emisión de reportes y otros dentro de Atención\sphinxhyphen{}1 agrupados por roles dentro del sistema.


\begin{savenotes}\sphinxatlongtablestart\begin{longtable}[c]{|*{11}{\X{1}{11}|}}
\hline
\sphinxstyletheadfamily &\sphinxstyletheadfamily &\sphinxstyletheadfamily 
Roles
&\sphinxstyletheadfamily &\sphinxstyletheadfamily &\sphinxstyletheadfamily &\sphinxstyletheadfamily &\sphinxstyletheadfamily &\sphinxstyletheadfamily &\sphinxstyletheadfamily &\sphinxstyletheadfamily \\
\hline\sphinxstyletheadfamily 
N
&\sphinxstyletheadfamily 
Acciones
&\sphinxstyletheadfamily 
Operador
&\sphinxstyletheadfamily 
Despachador
&\sphinxstyletheadfamily 
Médico
&\sphinxstyletheadfamily 
Coordinador
&\sphinxstyletheadfamily 
Asistente
&\sphinxstyletheadfamily 
Gerente
&\sphinxstyletheadfamily 
Director
&\sphinxstyletheadfamily 
Admin
&\sphinxstyletheadfamily \\
\hline
\endfirsthead

\multicolumn{11}{c}%
{\makebox[0pt]{\sphinxtablecontinued{\tablename\ \thetable{} \textendash{} proviene de la página anterior}}}\\
\hline
\sphinxstyletheadfamily &\sphinxstyletheadfamily &\sphinxstyletheadfamily 
Roles
&\sphinxstyletheadfamily &\sphinxstyletheadfamily &\sphinxstyletheadfamily &\sphinxstyletheadfamily &\sphinxstyletheadfamily &\sphinxstyletheadfamily &\sphinxstyletheadfamily &\sphinxstyletheadfamily \\
\hline\sphinxstyletheadfamily 
N
&\sphinxstyletheadfamily 
Acciones
&\sphinxstyletheadfamily 
Operador
&\sphinxstyletheadfamily 
Despachador
&\sphinxstyletheadfamily 
Médico
&\sphinxstyletheadfamily 
Coordinador
&\sphinxstyletheadfamily 
Asistente
&\sphinxstyletheadfamily 
Gerente
&\sphinxstyletheadfamily 
Director
&\sphinxstyletheadfamily 
Admin
&\sphinxstyletheadfamily \\
\hline
\endhead

\hline
\multicolumn{11}{r}{\makebox[0pt][r]{\sphinxtablecontinued{continué en la próxima página}}}\\
\endfoot

\endlastfoot

1
&
Visualizar listados de atenciones
&
\(\checkmark\)
&
\(\checkmark\)
&
\(\checkmark\)
&
\(\checkmark\)
&
\(\checkmark\)
&
\(\checkmark\)
&
\(\checkmark\)
&
\(\checkmark\)
&\\
\hline
2
&
Visualizar detalles de atenciones
&
\(\checkmark\)
&
\(\checkmark\)
&
\(\checkmark\)
&
\(\checkmark\)
&
\(\checkmark\)
&
\(\checkmark\)
&
\(\checkmark\)
&
\(\checkmark\)
&\\
\hline
3
&
Crear nueva atención
&
\(\checkmark\)
&
✗
&
✗
&
\(\checkmark\)
&
✗
&
\(\checkmark\)
&
\(\checkmark\)
&
\(\checkmark\)(*)
&\\
\hline
4
&
Editar información general de una atención
&
\(\checkmark\)
&
✗
&
✗
&
\(\checkmark\)
&
✗
&
\(\checkmark\)
&
\(\checkmark\)
&
\(\checkmark\)(*)
&\\
\hline
5
&
Editar datos personales de un afiliado
&
✗
&
✗
&
✗
&
✗
&
✗
&
✗
&
✗
&
\(\checkmark\)(*)
&\\
\hline
6
&
Editar pólizas de un afiliado
&
✗
&
✗
&
✗
&
✗
&
✗
&
✗
&
✗
&
\(\checkmark\)(*)
&\\
\hline
7
&
Editar historial de atenciones de un afiliado
&
✗
&
✗
&
✗
&
✗
&
✗
&
✗
&
✗
&
\(\checkmark\)(*)
&\\
\hline
8
&
Visualizar listado de servicios asociados a una atención
&
\(\checkmark\)
&
\(\checkmark\)
&
\(\checkmark\)
&
\(\checkmark\)
&
\(\checkmark\)
&
\(\checkmark\)
&
\(\checkmark\)
&
\(\checkmark\)
&\\
\hline
9
&
Crear un nuevo servicio
&
\(\checkmark\)
&
\(\checkmark\)
&
\(\checkmark\)
&
\(\checkmark\)
&
✗
&
\(\checkmark\)
&
\(\checkmark\)
&
\(\checkmark\)(*)
&\\
\hline
10
&
Visualizar detalles de un servicio
&
\(\checkmark\)
&
\(\checkmark\)
&
\(\checkmark\)
&
\(\checkmark\)
&
\(\checkmark\)
&
\(\checkmark\)
&
\(\checkmark\)
&
\(\checkmark\)
&\\
\hline
11
&
Editar información general de un servicio
&
\(\checkmark\)
&
\(\checkmark\)
&
\(\checkmark\)
&
\(\checkmark\)
&
✗
&
\(\checkmark\)
&
\(\checkmark\)
&
\(\checkmark\)(*)
&\\
\hline
12
&
Editar detalles médicos (diagnósticos y solicitudes) de un servicio OMT
&
✗
&
✗
&
\(\checkmark\)
&
✗
&
✗
&
✗
&
\(\checkmark\)
&
\(\checkmark\)(*)
&\\
\hline
13
&
Editar detalles médicos (diagnósticos y solicitudes) y operativos (tripulación y ruta) de un servicio AMD
&
✗
&
\(\checkmark\)
&
✗
&
✗
&
✗
&
✗
&
\(\checkmark\)
&
\(\checkmark\)(*)
&\\
\hline
14
&
Editar detalles médicos (diagnósticos) y operativos (tripulación y ruta) de un servicio Traslado
&
✗
&
\(\checkmark\)
&
✗
&
✗
&
✗
&
✗
&
\(\checkmark\)
&
\(\checkmark\)(*)
&\\
\hline
15
&
Editar detalles operativos (tripulación y ruta, nota de despacho) de un servicio EMD
&
✗
&
\(\checkmark\)
&
✗
&
✗
&&
✗
&
✗
&
\(\checkmark\)
&
\(\checkmark\)(*)
\\
\hline
16
&
Visualizar el flujo de trabajo asociado a servicios
&
\(\checkmark\)
&
\(\checkmark\)
&
\(\checkmark\)
&
\(\checkmark\)
&
\(\checkmark\)
&
\(\checkmark\)
&
\(\checkmark\)
&
\(\checkmark\)
&\\
\hline
17
&
Editar el flujo de trabajo asociado a un servicio OMT
&
✗
&
✗
&
\(\checkmark\)
&
✗
&
✗
&
✗
&
\(\checkmark\)
&
\(\checkmark\)(*)
&\\
\hline
18
&
Editar el flujo de trabajo asociado a un servicio AMD/Traslado/EMD
&
✗
&
\(\checkmark\)
&
✗
&
✗
&
✗
&
✗
&
\(\checkmark\)
&
\(\checkmark\)(*)
&\\
\hline
19
&
Crear un servicio sucesivo
&
\(\checkmark\)
&
\(\checkmark\)
&
\(\checkmark\)
&
\(\checkmark\)
&
✗
&
\(\checkmark\)
&
\(\checkmark\)
&
\(\checkmark\)(*)
&\\
\hline
20
&
Cerrar un servicio
&
✗
&
✗
&
✗
&
\(\checkmark\)
&
✗
&
\(\checkmark\)
&
\(\checkmark\)
&
\(\checkmark\)(*)
&\\
\hline
21
&
Cancelar un servicio OMT
&
✗
&
✗
&
\(\checkmark\)
&
✗
&
✗
&
✗
&
\(\checkmark\)
&
\(\checkmark\)(*)
&\\
\hline
22
&
Cancelar un servicio AMD/Traslado/EMD
&
✗
&
\(\checkmark\)
&
✗
&
✗
&
✗
&
✗
&
\(\checkmark\)
&
\(\checkmark\)(*)
&\\
\hline
23
&
Visualizar listado de afiliados
&
✗
&
✗
&
✗
&
✗
&
✗
&
✗
&
✗
&
\(\checkmark\)
&\\
\hline
24
&
Buscar afiliado
&
\(\checkmark\)
&
✗
&
✗
&
\(\checkmark\)
&
✗
&
\(\checkmark\)
&
\(\checkmark\)
&
\(\checkmark\)
&\\
\hline
25
&
Crear nuevo afiliado
&
\(\checkmark\)
&
✗
&
✗
&
\(\checkmark\)
&
✗
&
\(\checkmark\)
&
\(\checkmark\)
&
\(\checkmark\)(*)
&\\
\hline
26
&
Editar afiliado
&
✗
&
✗
&
✗
&
✗
&
✗
&
✗
&
✗
&
\(\checkmark\)(*)
&\\
\hline
27
&
Editar perfiles de usuario
&
✗
&
✗
&
✗
&
✗
&
✗
&
✗
&
✗
&
\(\checkmark\)(*)
&\\
\hline
28
&
Asignar roles
&
✗
&
✗
&
✗
&
✗
&
✗
&
✗
&
✗
&
\(\checkmark\)
&\\
\hline
29
&
Crear tripulaciones
&
✗
&
\(\checkmark\)
&
✗
&
✗
&
✗
&
✗
&
\(\checkmark\)
&
\(\checkmark\)(*)
&\\
\hline
30
&
Visualizar tripulaciones
&
✗
&
\(\checkmark\)
&
✗
&
\(\checkmark\)
&
✗
&
✗
&
\(\checkmark\)
&
\(\checkmark\)
&\\
\hline
31
&
Editar/eliminar tripulaciones
&
✗
&
\(\checkmark\)
&
✗
&
✗
&
✗
&
✗
&
\(\checkmark\)
&
\(\checkmark\)(*)
&\\
\hline
32
&
Generar reportes
&
✗
&
✗
&
✗
&
\(\checkmark\)
&
\(\checkmark\)
&
\(\checkmark\)
&
\(\checkmark\)
&
✗
&\\
\hline
33
&
Imprimir o generar PDF de ticket de datos de servicios
&
\(\checkmark\)
&
\(\checkmark\)
&
\(\checkmark\)
&
\(\checkmark\)
&
\(\checkmark\)
&
\(\checkmark\)
&
\(\checkmark\)
&
✗
&\\
\hline
34
&
Imprimir o generar PDF de ticket de nota de despacho de medicamentos
&
\(\checkmark\)
&
\(\checkmark\)
&
\(\checkmark\)
&
\(\checkmark\)
&
\(\checkmark\)
&
\(\checkmark\)
&
\(\checkmark\)
&
✗
&\\
\hline
35
&
Imprimir o generar PDF de ticket de tripulaciones
&
\(\checkmark\)
&
\(\checkmark\)
&
\(\checkmark\)
&
\(\checkmark\)
&
\(\checkmark\)
&
\(\checkmark\)
&
\(\checkmark\)
&
✗
&\\
\hline&\begin{itemize}
\item {} 
Solo bajo la autorización de Director.

\end{itemize}
&&&&&&&&&\\
\hline
\end{longtable}\sphinxatlongtableend\end{savenotes}


\chapter{Glosario}
\label{\detokenize{05glosario:glosario}}\label{\detokenize{05glosario:id1}}\label{\detokenize{05glosario::doc}}\begin{itemize}
\item {} 
\sphinxstylestrong{Administrador}: Rol de usuario del Sistema Atención\sphinxhyphen{}1, con permisos de acceso a la interfaz administrativa de la plataforma.

\item {} 
\sphinxstylestrong{Afiliado}: Persona que contrata o es beneficiario de una póliza de seguros. Atención\sphinxhyphen{}1 maneja dos tipos de afiliados: Titular (persona a nombre de quien figura la póliza), y Beneficiario/a (pesona que se incluye en la cobertura de la póliza)

\item {} 
\sphinxstylestrong{AMD}:Abreviatura de Atención Médica Domiciliaria, que es uno de los servicios gestionados por Atención\sphinxhyphen{}1 a través del cual un afiliado recibe servicios sanitarios a través de un médico en un lugar distinto de un centro médico.

\item {} 
\sphinxstylestrong{Atención}: Conjunto de servicios involucrados en una solución médica ofrecida a afiliados de las organizaciones aliadas de Atención\sphinxhyphen{}1.

\item {} 
\sphinxstylestrong{Base de datos}: Estructura digital que permite el registro, almacenamiento, acceso, modificación y eliminación de información y datos.

\item {} 
\sphinxstylestrong{Cliente}: Organizaciones, públicas o privadas, que prestan sus servicios de atención médica a través de Atención\sphinxhyphen{}1.

\item {} 
\sphinxstylestrong{Contratante}: Organización o persona que adquiere una póliza a través de cualquiera de las organizaciones aliadas de Atención\sphinxhyphen{}1.

\item {} 
\sphinxstylestrong{Coordinador}: Rol de usuario del Sistema Atención\sphinxhyphen{}1, con permisos de acceso a tareas de gestión de servicios y atenciones dentro de la plataforma Atención\sphinxhyphen{}1.

\item {} 
\sphinxstylestrong{Despachador}: Rol de usuario del Sistema Atención\sphinxhyphen{}1, con permisos de acceso a labores de gestión de tripulaciones, operaciones con traslados de afiliados, entregas de medicamentos y otros servicios ofrecidos por Atención\sphinxhyphen{}1 que involucren el traslado de medicamentos, personal y/o equipos.

\item {} 
\sphinxstylestrong{Director}: Rol de usuario del Atención\sphinxhyphen{}1, con permisos de acceso a la gestión integral de todas las operaciones de Atención\sphinxhyphen{}1, desde la creación de atenciones, hasta el cierre y cancelación de servicios y atenciones y la generación de reportes.

\item {} 
\sphinxstylestrong{Destino}: Dirección en la que se prevé termine un servicio que involucre un traslado o el desplazamiento de una unidad vehicular.

\item {} 
\sphinxstylestrong{Diagnóstico}: Procedimiento médico a través del cual se identifica cualquier enfermedad, sindrome o similar.

\item {} 
\sphinxstylestrong{EMD}: Abreviatura de Entrega de Medicamentos Domiciliaria, que es parte de los servicios gestionados por Atención\sphinxhyphen{}1 y que permite el envío de medicamentos solicitados por cualquier afiliado de pólizas con estos beneficios, al domicilio que éste indique.

\item {} 
\sphinxstylestrong{Gerente}: Rol de usuario de Atención\sphinxhyphen{}1, con permisos de acceso a la gestión de las operaciones y supervisión en el mismo. Este rol es el encargado de la solicitud y generación de reportes sobre estatus en tiempo real de las atenciones y servicios y creación de tripulación entre otros.

\item {} 
\sphinxstylestrong{Indicaciones}: Conjunto de instrucciones detalladas sobre medicamentos indicados según diagnóstico, y presentadas en un récipe médico a la persona afiliada con la información necesaria para la realización del tratamiento médico indicado por el personal médico.

\item {} 
\sphinxstylestrong{Médico}: Rol de usuario de Atención\sphinxhyphen{}1, con permisos de atención de un servicio OMT y creación de servicios sucesivos.

\item {} 
\sphinxstylestrong{Medicamento}: Sustancia que es utilizada en la prevención o tratamiento de alguna enfermedad, bien sea eliminando sus efectos en el organismo, o mitigándolos.

\item {} 
\sphinxstylestrong{OMT}: Abreviatura de Orientación Médica Telefónica, que es parte de los servicios gestionados por Atención\sphinxhyphen{}1 y que implica la atención telefónica de afiliados en la solución de sus consultas o derivación hacia otros servicios.

\item {} 
\sphinxstylestrong{Organización}: Entidad colectiva, de tipo Contratante, Cliente o Externo, que presta o recibe servicios de salud a través de la plataforma Atención\sphinxhyphen{}1.

\item {} 
\sphinxstylestrong{Proveedor Externo}: Empresas de traslados y servicios médicos que sirve de apoyo en los servicios AMD y EMD cuando éstos se prestan fuera del ámbito de la Gran Caracas.

\item {} 
\sphinxstylestrong{Operador}: Rol de usuario del Sistema Atención\sphinxhyphen{}1, con permisos de acceso a tareas de toma y registro de atenciones y solicitudes de servicios dentro de la plataforma Atención\sphinxhyphen{}1.

\item {} 
\sphinxstylestrong{Plan}: Conjunto de servicios y características disponibles según las póliza contratada por el afiliado.

\item {} 
\sphinxstylestrong{Póliza}: Número que identifica al documento que recoge las condiciones generales del seguro contratado, y las condiciones particulares así como los suplementos o apéndices emitidos para complementarla o modificarla.

\item {} 
\sphinxstylestrong{Reporte}: Documento digital obtenido a través del sistema Atención\sphinxhyphen{}1 que permite plasmar de forma resumida, a través de gráficos y tablas, información relativa a servicios, clientes y afiliados atendidos, fechas y horas más utilizadas por los afiliados al sistema, para un periodo de tiempo.

\item {} 
\sphinxstylestrong{Servicio}: Prestación sanitaria que se ofrece a afiliados dentro de la plataforma Atención\sphinxhyphen{}1.

\item {} 
\sphinxstylestrong{Síntoma}: Alteración de las condiciones normales del organismo, que permite evidenciar la presencia de una enfermedad, su determinación ayuda a establecer la naturaleza de la enfermedad.

\item {} 
\sphinxstylestrong{TLD}: Abreviatura de Traslado, que es parte de los servicios gestionados por Atención\sphinxhyphen{}1 y que implica el desplazamiento de un vehículo de Venemergencia o de algún proveedor externo, a fin de transportar a un afiliado desde un domicilio hasta un centro asistencial.

\item {} 
\sphinxstylestrong{Tripulación}: Conjunto de profesionales de la salud y logística que están involucrados en servicios ofrecidos a afiliados, y que se trasladan en un vehículo automotor con el propósito de ofrecer algún servicio de los disponibles en Atención\sphinxhyphen{}1.

\item {} 
\sphinxstylestrong{Usuario}: Nombre genérico utilizado para designar el conjunto de roles disponibles en Atención\sphinxhyphen{}1 al personal de Venemergencia para ofrecer servicios médicos a través de la plataforma.

\end{itemize}


\chapter{Acerca de Atención\sphinxhyphen{}1}
\label{\detokenize{06about:acerca-de-atencion-1}}\label{\detokenize{06about::doc}}

\chapter{Indices y tablas}
\label{\detokenize{index:indices-y-tablas}}\begin{itemize}
\item {} 
\DUrole{xref,std,std-ref}{genindex}

\item {} 
\DUrole{xref,std,std-ref}{modindex}

\item {} 
\DUrole{xref,std,std-ref}{search}

\end{itemize}



\renewcommand{\indexname}{Índice}
\printindex
\end{document}